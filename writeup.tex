% CS267 - Project Proposal - Kolb & Gardner

\documentclass[american]{article}
\usepackage[american]{babel}

\usepackage[margin=2.5cm]{geometry}
\usepackage[backend=biber, style=authoryear]{biblatex}
\addbibresource{proj_refs.bib}
\DeclareLanguageMapping{american}{american-apa}
\usepackage{csquotes}
\usepackage{graphicx}
\usepackage{caption}
\usepackage{subcaption}
\usepackage{wrapfig}

\begin{document}

\title{\vspace{-3em}CS267 - Project Proposal}
\author{Jack Kolb \& Michael Gardner}
\date{\today}
\maketitle
\vspace{-4em}

\section{Introduction}
Geotechnics is a branch of civil engineering that is concerned with the study and modification of soil and rock. For example, engineers and scientists are often interested in studying jointed rock masses -- a body of rock that is broken into discrete blocks by cracks, fractures, and other discontinuities in the rock field. The behavior of these blocks can then be modeled using well-known discrete element methods. However, one of the more challenging aspects in this kind of analysis and simulation is the identification of the discrete blocks that will be used to model the actual rock mass. In three-dimensional space, identification of discrete polyhedral blocks is extremely tedious because algorithms need to keep track of the vertices, edges and faces of each block as well as the hierarchy of how these features are connected \parencite{Slicing}.

\textcite{Collision} propose a new algorithm that requires knowledge only of the faces of each polyhedral block. The space occupied by each polyhedron can be defined as a set of linear inequalities such that determining joint intersections and contact can be solved using techniques from convex optimization, specifically linear programming. As such, only information on the faces of the polyhedra and discontinuities are required to determine blocking and contact. This is a much simpler alternative compared to previous approaches that relies on simpler and more robust data structures. After expressing each region as a set of linear inequalities, the algorithm iteratively considers each known discontinuity and divides each block that intersects the discontinuity into two child blocks. While these child blocks can be treated independently in future iterations, presenting an excellent opportunity for parallelism, the algorithm instead takes a serial approach.

\section{Implementation on Apache Spark}
The problem of identifying the discrete blocks in a joined rock mass lends itself to being solved with Apache Spark, since a large initial data set will have to be queried multiple times to determine fracture orientations, and we can divide the domain into independent regions that are analyzed independently and in parallel. The number of rock blocks in a discrete element model can be on the order of  several thousand, each of which needs to be explicitly defined before any simulation can be done. The process of defining this discretization can be massively time consuming and computationally expensive, but as explained above, the algorithm is amenable to parallel execution.

The initial goal of this project would be to take a simple input geometry, a set of discontinuities, and an assumed failure plane and output a set of blocks that could be used in a discrete element analysis. We can then measure the performance of our parallel implementation and compare this to the original serial version. If time permits, the project could be extended to have the initial input be a full-scale LIDAR survey and information on a likely failure plane. The program would then determine fracture orientations and prevalence before starting the rock slicing computations.

The algorithm described in \textcite{Collision} relies heavily on efficient solving of linear programs as a basic primitive. Rather than implement our own LP solver, which could be a project in and of itself, we plan to use an existing framework such as GLPK or lp\_solve. In fact, Spark's own MLlib framework may be suitable for this task.

%\clearpage
\nocite{*}
\printbibliography

\end{document}
